\documentclass{article}
\usepackage[utf8x]{inputenc}
\usepackage[T1, T2A]{fontenc}
\usepackage[russian]{babel}
\usepackage{amsmath}
\usepackage{amssymb}
\setlength\parindent{0pt}
\usepackage[parfill]{parskip}
\pagenumbering{gobble}

\begin{document}
В задаче требуется найти число различных наборов $a,b,c,d,e,f$ таких, что выполнено
$$\begin{pmatrix}1&2&3&4&5&6\\b&c&a&e&f&d\end{pmatrix} = \begin{pmatrix}1&2&3&4&5&6\\a&b&c&d&e&f\end{pmatrix} \begin{pmatrix}1&2&3&4&5&6\\2&3&1&5&6&4\end{pmatrix}.$$
Заметим, что, например, утверждение $a=1$ полностью задает весь цикл ($a=1$, $b=2$, $c=3$). Подставляя другие значения $a$ убеждаемся, что на местах $a,b,c$ может быть либо циклическая перестановка $1,2,3$, либо циклическая перестановка $4,5,6$. Проведя аналогичные рассуждения для $d,e,f$ получим, что число всех возможных наборов равно $2 \cdot 3^2 = 18$.\\
\textit{Примечание.} Точно так же можно доказать и более общее утверждение. Если перестановка состоит из циклов, различные длины которых равны $n_1, n_2, \ldots, n_k$ и каждая длина $n_i$ встречается $m_i$ раз, то число перестановок, которые с ней коммутируют, равно
$$\prod\limits_{i=1}^k m_i! \: n_i^{m_i}.$$
\end{document}

\documentclass{article}
\usepackage[utf8x]{inputenc}
\usepackage[T1, T2A]{fontenc}
\usepackage[russian]{babel}
\usepackage{amsmath}
\usepackage{amssymb}
\setlength\parindent{0pt}
\usepackage[parfill]{parskip}
\pagenumbering{gobble}

\begin{document}
Предположим, что у такой матрицы есть рациональное целое собственное значение в виде несократимой дроби $p/m$. Подставим его в характеристическое уравнение на собственные значения матрицы:
$$\left( \frac{p}{m} \right)^n + a_{n-1} \left( \frac{p}{m} \right)^{n-1} + \cdots + a_0 = 0.$$
Домножим все члены этого уравнения на $m^{n-1}$:
$$\frac{p^n}{m} + a_{n-1} p^{n-1} + \cdots + a_0 m^{n-1} = 0.$$
Заметим, что в этом уравнении только одно нецелое слагаемое $\frac{p^n}{m}$, чего быть не может.
\end{document}

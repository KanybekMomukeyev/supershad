\documentclass{article}
\usepackage[utf8x]{inputenc}
\usepackage[T1, T2A]{fontenc}
\usepackage[russian]{babel}
\usepackage{amsmath}
\usepackage{amssymb}
\setlength\parindent{0pt}
\usepackage[parfill]{parskip}
\pagenumbering{gobble}

\begin{document}
На круговой дороге стоят канистры с бензином. Есть машина с известным расходом топлива и пустым баком неограниченной емкости. 
За $O(n)$ операций выясните, от какой канистры надо начать, чтобы, собирая топливо, проехать всю дорогу и не остановиться пустым (или сказать, что это невозможно).
\end{document}

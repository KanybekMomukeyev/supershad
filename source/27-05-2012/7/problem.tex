\documentclass{article}
\usepackage[utf8x]{inputenc}
\usepackage[T1, T2A]{fontenc}
\usepackage[russian]{babel}
\usepackage{amsmath}
\usepackage{amssymb}
\setlength\parindent{0pt}
\usepackage[parfill]{parskip}
\pagenumbering{gobble}

\begin{document}
Есть круговая трасса, на которой в некоторых местах стоят бензоколонки. Расстояние между ними и количество бензина 
на каждой бензоколонке известны. Имеется также машина с постоянным и известным расходом топлива. Предложите алгоритм, работающий за $O(n)$ по времени, который позволяет найти ту бензоколонку, начиная с которой можно проехать всю трассу, или сказать, что такой нет.
\end{document}

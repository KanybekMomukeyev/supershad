\documentclass{article}
\usepackage[utf8x]{inputenc}
\usepackage[T1, T2A]{fontenc}
\usepackage[russian]{babel}
\usepackage{amsmath}
\usepackage{amssymb}
\setlength\parindent{0pt}
\usepackage[parfill]{parskip}
\pagenumbering{gobble}

\begin{document}
Из $P(x) > 0$ следует, что у многочлена нет вещественных корней. Поскольку коэффициенты многочлена действительные, для любого комплексного корня $x_i$ комплексно-сопряженное число $\overline{x_i}$ также является корнем. Поэтому многочлен можно представить в виде:
$$P(x) = (x - x_1)(x - \overline{x_1}) \cdots (x-x_k)(x - \overline{x_k}).$$
Попарно раскрывая скобки получим следующий вид выражения:
$$P(x) = (x^2 + b_1x + c_1) \cdots (x^2 + b_kx + c_k).$$
Выделим полный квадрат в каждом из квадратичных множителей:
$$P(x) = ((x + b_1/2)^2 + q_1) \cdots ((x + b_k/2)^2 + q_k).$$
Заметим, что $q_i > 0$ для любого $i$, поскольку иначе у выражения $(x + b_i/2)^2 + q_i$ были бы действительные корни. Значит, мы можем написать $q_i = y_i^2$, где $y_i$ --- некоторое действительное число. Тогда
$$P(x) = ((x + b_1/2)^2 + (y_1)^2) \cdots ((x + b_k/2)^2 + (y_k)^2).$$
Как видим, при раскрытии скобок мы получим сумму квадратов многочленов с действительными коэффициентами, что и требовалось доказать.
\end{document}

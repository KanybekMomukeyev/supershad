\documentclass{article}
\usepackage[utf8x]{inputenc}
\usepackage[T1, T2A]{fontenc}
\usepackage[russian]{babel}
\usepackage{amsmath}
\usepackage{amssymb}
\setlength\parindent{0pt}
\usepackage[parfill]{parskip}
\pagenumbering{gobble}

\begin{document}
Для определенности положим
$$K_{ij} = \begin{cases} 1,&\text{если } i\text{-й подписан на }j\text{-ого;}\\ 0,&\text{иначе.}\end{cases}$$
Заметим, что если $K_{ij} = 1$, то $i$-ый не может быть знаменитостью, а если $K_{ij} = 0$, то $j$-ый не может быть знаменитостью. Таким образом, за 
одну проверку можно исключить одного человека из кандидатов в знаменитости.\\
Сначала пусть $s=1$, а $l$ пробегает значения от $2$ до $n$. Если в какой-то момент $K_{sl}=1$, то приравниваем $s=l$. Тогда значение 
$s$ после последней проверки --- номер единственного оставшегося кандидата. Чтобы проверить, является ли этот кандидат знаменитостью, нужно провести еще 
$n-1$ проверок, знают ли его остальные, и $n-1$ проверок, знает ли он остальных. Всего будет проведено $3(n-1)$ проверок, следовательно, сложность по времени --- $O(n)$. 
Поскольку мы использовали только $2$ переменные, сложность по памяти --- $O(1)$.
\end{document}

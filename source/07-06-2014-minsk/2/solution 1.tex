\documentclass{article}
\usepackage[utf8x]{inputenc}
\usepackage[T1, T2A]{fontenc}
\usepackage[russian]{babel}
\usepackage{amsmath}
\usepackage{amssymb}
\setlength\parindent{0pt}
\usepackage[parfill]{parskip}
\pagenumbering{gobble}

\begin{document}
Рассмотрим функцию $z = f(x,y)+100 = (x-6)^2 + (y+8)^2$. Понятно, что эта функция достигает минимума и максимума в тех же точках, что и $f(x,y)$. Видно, что 
$z$ постоянна на окружности с центром в точке $(6, -8)$ и увеличивается с увеличением радиуса окружности, поэтому минимум достигается в ближней точке касания 
этой окружности с окружностью $x^2 + y^2 = 5^2$, а максимум --- в дальней.\\
Заметим, что через точки касания окружностей проходит прямая, соединяющая центры окружностей. Поэтому, достаточно найти точки пересечения окружности $x^2 + y^2 = 5^2$ с 
прямой $y = -\frac{8x}{6}$. Получим $(3,-4)$ и $(-3,4)$. Им соответствуют значения $z=25$ и $z=225$.
Тогда $\min (f(x,y)) = -75$, а $\max (f(x,y)) = 125$.
\end{document}

\documentclass{article}
\usepackage[utf8x]{inputenc}
\usepackage[T1, T2A]{fontenc}
\usepackage[russian]{babel}
\usepackage{amsmath}
\usepackage{amssymb}
\setlength\parindent{0pt}
\usepackage[parfill]{parskip}
\pagenumbering{gobble}

\begin{document}
Движение роботов периодичное с периодом $T = 2L/1 = 2L$, поэтому вместо интервала $t$ мы можем рассматривать интервал $t - [t/T]T$.\\
(a) Заметим, что если мы не будем различать роботов, то их соударения не будут менять ничего. Поэтому мы просто можем считать, что каждый робот соударяется только со стенками. Тогда его положение легко найти за $O(1)$. На выходе получим набор координат, некоторые из которых могут повторяться. Убрать повторы можно сортировкой и последующим проходом по набору. Итоговая сложность --- $O(n \log n)$.\\
(б) Порядок номеров роботов не меняется в процессе движения. Поэтому, если мы возьмем набор координат из предыдущего пункта (повторы не удаляем) и пронумеруем их в том же порядке, в каком они были пронумерованы в начале движения, мы получим ответ на данных пункт. Для этого можно отсортировать координаты по возрастанию или убыванию (в начальный момент и в момент времени $t$). Итоговая сложность --- $O(n \log n)$. 
\end{document}

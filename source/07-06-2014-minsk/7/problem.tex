\documentclass{article}
\usepackage[utf8x]{inputenc}
\usepackage[T1, T2A]{fontenc}
\usepackage[russian]{babel}
\usepackage{amsmath}
\usepackage{amssymb}
\setlength\parindent{0pt}
\usepackage[parfill]{parskip}
\pagenumbering{gobble}

\begin{document}
В коридоре длины $L$ находятся $n$ роботов, $i$-й из которых изначально расположен в позиции $x_i$ 
(все позиции различны $0 \leqslant x_i \leqslant L$). Все роботы движутся с единичной скоростью вдоль коридора. $i$-й робот движется со скоростью $v_i\; (\pm 1)$. При столкновении робота с границей коридора или с другим роботом направление вектора скорости робота меняется на противоположное.\\
Прошло $t$ единиц времени...\\
(a) Требуется найти множество точек, в которых находятся роботы (без учета порядка: не важно, какой робот находится в какой точке; точки в множестве не должны повторятся).\\
(б) Для каждого робота $i$ необходимо указать его финальное положение $y_i$ в коридоре.\\
Предложите эффективный алгоритм решения этих задач.
\end{document}

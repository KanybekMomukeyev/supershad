\documentclass{article}
\usepackage[utf8x]{inputenc}
\usepackage[T1, T2A]{fontenc}
\usepackage[russian]{babel}
\usepackage{amsmath}
\usepackage{amssymb}
\setlength\parindent{0pt}
\usepackage[parfill]{parskip}
\pagenumbering{gobble}

\begin{document}
Назовем элемент прямоугольной матрицы \textit{седлом}, если он является наибольшим в своей строке и наименьшим в своем столбце или наоборот. Придумайте алгоритм, за $O(nm)$ операций находящий все седла в матрице $\text{A[1\,.\,.\,n][1\,.\,.\,m]}$, использующий не более $O(n+m)$ памяти и не более $1$ раза обращающийся к каждому элементу $\text{A}$ (можете считать, что элемент $\text{A[i][j]}$ превращается в $\text{NaN}$ сразу после вызова $\text{A[i][j]}$). Считайте, что все элементы матрицы различны.
\end{document}

\documentclass{article}
\usepackage[utf8x]{inputenc}
\usepackage[T1, T2A]{fontenc}
\usepackage[russian]{babel}
\usepackage{amsmath}
\usepackage{amssymb}
\setlength\parindent{0pt}
\usepackage[parfill]{parskip}
\pagenumbering{gobble}

\begin{document}
Немного преобразуем выражение:
$$\sum_{i=0}^n \frac{1}{2^{m+i+1}} C_{m+i}^i + \sum_{i=0}^m \frac{1}{2^{n+i+1}} C_{n+i}^i = \frac{1}{2^{m+n+1}} \left( \sum_{i=0}^n 2^{n-i} C_{m+i}^i + \sum_{i=0}^m 2^{m-i} C_{n+i}^i \right).$$
Знаменатель можно проинтерпретировать как количество последовательностей из $0$ и $1$ длины $m+n+1$. Пусть $X$ -- множество последовательностей, в которых хотя бы $n+1$
единица, а $Y$ -- множество последовательностей, в которых хотя бы $m+1$ ноль. Ясно, что $X \cap Y = \varnothing$, а $X \cup Y$ -- множество всех последовательностей.
Найдём мощность множества $X$. Пусть $(n + 1)$-я единица стоит на месте с номером $n + 1 +i$, где $i \in \{0, \ldots , m\}$. Тогда первые $n + i$ членов последовательности полностью
определяются выбором позиций, на которых стоят единицы (это можно сделать $C_{n+i}^n=C_{n+i}^i$ способами), а последние $n-i$ могут быть любыми, то есть их можно выбрать $2^{n-i}$ способами.
Таким образом, число последовательностей, содержащих хотя бы $n+1$ единицу, равно $\sum\limits_{i=0}^m 2^{m-i} C_{n+i}^i$.
Аналогично число последовательностей, содержащих хотя бы $m+1$ ноль, равно $\sum\limits_{i=0}^n 2^{n-i} C_{m+i}^i$. А поскольку каждая последовательность принадлежит к одному и только
одному из этих видов, то мы делаем вывод, что
$$\sum_{i=0}^n 2^{n-i} C_{m+i}^i + \sum_{i=0}^m 2^{m-i} C_{n+i}^i = 2^{m+n+1}.$$
Таким образом,
$$\sum_{i=0}^n \frac{1}{2^{m+i+1}} C_{m+i}^i + \sum_{i=0}^m \frac{1}{2^{n+i+1}} C_{n+i}^i = 1.$$
\end{document}

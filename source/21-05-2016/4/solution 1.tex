\documentclass{article}
\usepackage[utf8x]{inputenc}
\usepackage[T1, T2A]{fontenc}
\usepackage[russian]{babel}
\usepackage{amsmath}
\usepackage{amssymb}
\setlength\parindent{0pt}
\usepackage[parfill]{parskip}
\pagenumbering{gobble}

\begin{document}
Рассмотрим оператор, который вычисляет среднее значение функции $f(\vec{r})$ на окружности радиуса $1$ и вычитает значение функции в центре данной окружности:
$$(Tf) (\vec{r}) = f(\vec{r}) - \int\limits_{0}^{1} f(\vec{r} + \vec{\tau} (s)) ds,$$
где $\vec{\tau} (s) = (\begin{array}{cc} \cos (2\pi s) & \sin (2\pi s) \end{array})^\mathrm{T}$.
Найдем, как действует этот оператор при преобразовании Фурье $F$:
$$T_{F} \varphi (\vec{\omega}) = F T F^{-1} \varphi (\vec{\omega}) = \left(1 - \int\limits_{0}^{1} e^{i \vec{\tau} (s) \cdot \vec{\omega}} ds \right) \varphi (\vec{\omega}) = g(\vec{\omega}) \varphi (\vec{\omega}),$$
где $\varphi (\vec{\omega})$ принадлежит пространству обобщенных функций. Мы получили, что $T_F$ является оператором умножения на гладкую функцию $g(\vec{\omega})$. Найдем нули этой функции:
$$g(\vec{\omega}) =  1 - \int\limits_{0}^{1} e^{i \vec{\tau} (s) \cdot \vec{\omega}} ds = 1 - \int\limits_{0}^{1} e^{i |\vec{\omega}| (\cos \Phi \cos (2\pi s) +\sin \Phi  \sin (2\pi s)}ds = $$
$$ = 1 - \int\limits_{0}^{1} e^{i |\vec{\omega}| \cos (\Phi - 2\pi s)}ds = 1 - \int\limits_{0}^{1} e^{i |\vec{\omega}| \cos (2\pi s)}ds = $$
$$ = 1 - \int\limits_{0}^{1/2} e^{i |\vec{\omega}| \cos (2\pi s)} - \int\limits_{0}^{1/2} e^{-i |\vec{\omega}| \cos (2\pi s)}ds = 1 - 2 \int\limits_{0}^{1/2} \cos(|\vec{\omega}| \cos (2\pi s))ds.$$
За $\Phi$ мы обозначили угол между декартовыми компонентами $\vec{\omega}$. В дальнейшем он исчезает, поскольку интеграл берется по всему периоду периодической функции. Произведем замену $u = \cos (2\pi s)$. Тогда
$$g(\vec{\omega}) = 1 + \frac{1}{\pi} \int_{1}^{-1} \frac{\cos(|\vec{\omega}| u) du}{\sqrt{1 - u^2}} = 1 - \frac{2}{\pi} \int_{0}^{1} \frac{\cos(|\vec{\omega}| u) du}{\sqrt{1 - u^2}} = $$
$$ = \frac{2}{\pi} \int\limits_0^1 \frac{(1 - \cos(|\vec{\omega}| u) du}{\sqrt{1 - u^2}}.$$
Отсюда видно, что $g(\vec{\omega})$ имеет единственный нуль в точке $\vec{\omega} = 0$. Найдем порядок этого нуля. Нетрудно разложить $g(\vec{\omega})$ в ряд Маклорена с остаточным членом в форме Пеано:
$$g(\vec{\omega}) = \frac{2}{\pi} \int\limits_0^1 \left(\frac{\vec{\omega}^2u^2}{2} + O(\vec{\omega}^2u^2) \right) \frac{du}{\sqrt{1 - u^2}} = \frac14 \vec{\omega}^2 + O(\vec{\omega}^2).$$
Таким образом, порядок нуля равен двум. Отсюда следует, что $\text{ker}\, (T_F)$ состоит из линейной оболочки $\delta(\vec{\omega})$ и $\delta'(\vec{\omega})$. При обратном преобразовании Фурье эта оболочка соответствует множеству функций вида $f = Ax + By + C$. Поскольку только константа из них ограничена, утверждение задачи доказано.
\end{document}

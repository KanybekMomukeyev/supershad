\documentclass{article}
\usepackage[utf8x]{inputenc}
\usepackage[T1, T2A]{fontenc}
\usepackage[russian]{babel}
\usepackage{amsmath}
\usepackage{amssymb}
\setlength\parindent{0pt}
\usepackage[parfill]{parskip}
\pagenumbering{gobble}

\begin{document}
В мишень, которая представляет собой прямоугольник размера $3\times 2$, стреляют из пистолета.
Известно, что отклонение пули от точки, на которую нацелен пистолет, произвольно, но не превышает $0,1$ по любому направлению, параллельному сторонам прямоугольника.
Стрелок целится в произвольную точку мишени. С какой вероятностью он попадет в мишень?
\end{document}

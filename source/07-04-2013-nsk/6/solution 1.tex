\documentclass{article}
\usepackage[utf8x]{inputenc}
\usepackage[T1, T2A]{fontenc}
\usepackage[russian]{babel}
\usepackage{amsmath}
\usepackage{amssymb}
\setlength\parindent{0pt}
\usepackage[parfill]{parskip}
\pagenumbering{gobble}

\begin{document}
Пронумеруем бензоколонки по порядку начиная с произвольной. Пусть теперь $g[i]$ --- изменение количества топлива после заправки в $i$-ой бензоколонки и прохождения дороги после нее. Тогда если $\sum\limits_{i=1}^n g[i] < 0$ такой бензоколонки нет, поскольку в таком случае количество топлива в бензоколонках меньше, чем требуется для прохождения всех дорог. Покажем, что если $\sum\limits_{i=1}^n g[i] \geqslant 0$ такая бензоколонка существует всегда. Рассмотрим частичные суммы $h[k] = \sum\limits_{i=1}^k g[i]$. Пусть минимальная частичная сумма достигается в индексе $j$ и равна $h[j]$. Тогда если $h[j] \geqslant 0$, то задача решена. Если же нет --- начнем движение с бензоколонки, следующей за $j$. В таком случае частичные суммы всегда неотрицательны, поскольку в противном случае $j$ не был элементом, в котором достигается минимальная частичная сумма. Сложность по времени --- $O(n)$. 
\end{document}

\documentclass{article}
\usepackage[utf8x]{inputenc}
\usepackage[T1, T2A]{fontenc}
\usepackage[russian]{babel}
\usepackage{amsmath}
\usepackage{amssymb}
\setlength\parindent{0pt}
\usepackage[parfill]{parskip}
\pagenumbering{gobble}

\begin{document}
В неориентированном графе без петель и кратных ребер $2n$ вершин и $n^2+1$ ребро.
Треугольником в графе называется фигура, состоящая из трех вершин и трех соединяющих их ребер. Докажите, что в этом графе найдутся два треугольника с общим ребром.
\end{document}

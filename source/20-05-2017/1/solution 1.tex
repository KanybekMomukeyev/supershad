\documentclass{article}
\usepackage[utf8x]{inputenc}
\usepackage[T1, T2A]{fontenc}
\usepackage[russian]{babel}
\usepackage{amsmath}
\usepackage{amssymb}
\setlength\parindent{0pt}
\usepackage[parfill]{parskip}
\pagenumbering{gobble}

\begin{document}
Так как $A$ --- симметричная и положительно определена, то она может быть представлена в виде $A=C^TC$. Поэтому
$$
X^TC^TCX=(CX)^TCX=M^TM,\\
$$
отсюда, учитывая что $m_{ij}=\sum\limits_{k=1}^{n}c_{ik}x_{ki}$ получаем:
$$
tr(X^TC^TCX)=\sum\limits_{k=1}^{n}m_{ii}^2=\sum\limits_{k=1}^{n}(\sum\limits_{r=1}^{n}c_{ik}x_{ki})^2,
$$
то есть $tr>0$ для любых $X>0$.

\end{document}

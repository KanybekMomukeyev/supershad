\documentclass{article}
\usepackage[utf8x]{inputenc}
\usepackage[T1, T2A]{fontenc}
\usepackage[russian]{babel}
\usepackage{amsmath}
\usepackage{amssymb}
\setlength\parindent{0pt}
\usepackage[parfill]{parskip}
\pagenumbering{gobble}

\begin{document}
Все числа от $1$ до $n=2^k - 1$ записаны неизвестным нам образом
в полном бинарном дереве высоты $k$. Будем говорить, что число $t$ лежит между числами $i$ и $j$
в этом дереве, если при удалении $t$ из дерева $i$ и $j$ оказываются в разных компонентах. Предложите алгоритм, определяющий, что за число находится в корне дерева за $O(n \log n)$ операций с помощью запросов вида <<Лежит ли $t$ между 
$i$ и $j$?>>.
\end{document}

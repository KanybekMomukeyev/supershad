Пространство будет иметь размерность $n^2$(грубо говоря вытянем все столбцы в один).
Выберем базис следующим образом:

<ul>
<li> $n^2-1$ - матриц, у которых на $i,j$ месте
1, на остальных 0, $i,j \in \overline{1,n}, (i,j)\neq(n,1)$;
<li> одна матрица - матрица из всех едениц.
</ul>
Занумеруем векторы базиса следующим образом: матрица из всех единиц будет иметь
номер $n^2-n$, остальные векторы $(i-1)\cdot n+j$, где $i$ и $j$ - позиция ненулевого
элемента.
В таком базисе у верхнетреугольных матриц координата c номером $n^2-n$ будет
всегда нулевой. Из формулы скалярного произведения $$<a,b>=\sum_{i=1}^{n^2}a_ib_i,$$
$a_i,b_i$ - координаты в выбранном базисе, следует, что еденичная матрица
 ортогональна любой верхнетреугольной матрице.

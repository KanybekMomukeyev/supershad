\documentclass{article}
\usepackage[utf8x]{inputenc}
\usepackage[T1, T2A]{fontenc}
\usepackage[russian]{babel}
\usepackage{amsmath}
\usepackage{amssymb}
\setlength\parindent{0pt}
\usepackage[parfill]{parskip}
\pagenumbering{gobble}

\begin{document}
Пусть $k$ --- количество классификаторов. Тогда все возможные использования классификаторов можно представить в виде булевой матрицы $20\times k$,
в которой элемент $c_{ij}$ равен $1$, если для решения $i$-й задачи классификации используется $j$-й классификатор.
По условию задачи в каждой строке такой матрицы должно быть ровно $5$ единиц, а любая пара единиц $(j_1,j_2)$ может встречаться только в одной строке.
Нетрудно понять, что $k \geqslant 21$. В самом деле, в каждой строке присутствует $\frac{5\cdot 4}{2} = 10$ уникальных пар классификаторов.
Значит, число различных пар классификаторов, которые применялись для решения задач, равно $10\cdot 20 = 200$. С другой стороны, общее число пар классификаторов равно $\frac{k(k-1)}{2}$.
Очевидно, должно выполняться условие $200 \leqslant \frac{k(k-1)}{2}$. Следовательно, $k \geqslant 21$. Приведем пример матрицы $21\times 21$, у которой любая из подматриц $20\times 21$ удовлетворяет условию задачи: 
\setcounter{MaxMatrixCols}{21}
$$\begin{matrix}
0&0&0&0&0&0&0&0&0&0&0&0&0&0&0&0&1&1&1&1&1\\
0&0&0&0&0&0&0&0&0&0&0&0&1&1&1&1&0&0&0&0&1\\
0&0&0&0&0&0&0&0&1&1&1&1&0&0&0&0&0&0&0&0&1\\
0&0&0&0&1&1&1&1&0&0&0&0&0&0&0&0&0&0&0&0&1\\
1&1&1&1&0&0&0&0&0&0&0&0&0&0&0&0&0&0&0&0&1\\
0&0&0&1&0&0&0&1&0&0&0&1&0&0&0&1&0&0&0&1&0\\
0&0&1&0&0&0&1&0&0&0&1&0&0&0&1&0&0&0&0&1&0\\
0&1&0&0&0&1&0&0&0&1&0&0&0&1&0&0&0&0&0&1&0\\
1&0&0&0&1&0&0&0&1&0&0&0&1&0&0&0&0&0&0&1&0\\
1&0&0&0&0&1&0&0&0&0&1&0&0&0&0&1&0&0&1&0&0\\
0&1&0&0&1&0&0&0&0&0&0&1&0&0&1&0&0&0&1&0&0\\
0&0&1&0&0&0&0&1&1&0&0&0&0&1&0&0&0&0&1&0&0\\
0&0&0&1&0&0&1&0&0&1&0&0&1&0&0&0&0&0&1&0&0\\
0&0&1&0&1&0&0&0&0&1&0&0&0&0&0&1&0&1&0&0&0\\
0&0&0&1&0&1&0&0&1&0&0&0&0&0&1&0&0&1&0&0&0\\
1&0&0&0&0&0&1&0&0&0&0&1&0&1&0&0&0&1&0&0&0\\
0&1&0&0&0&0&0&1&0&0&1&0&1&0&0&0&0&1&0&0&0\\
0&1&0&0&0&0&1&0&1&0&0&0&0&0&0&1&1&0&0&0&0\\
1&0&0&0&0&0&0&1&0&1&0&0&0&0&1&0&1&0&0&0&0\\
0&0&0&1&1&0&0&0&0&0&1&0&0&1&0&0&1&0&0&0&0\\
0&0&1&0&0&1&0&0&0&0&0&1&1&0&0&0&1&0&0&0&0
\end{matrix}$$
Построить такую матрицу на удивление нетрудно. Пусть изначально у нас все нули. Идем по матрице справа налево и сверху вниз и ставим $1$, если мы можем это сделать (то есть если такой пары еще нет; для первой единицы в строке --- если она еще не образует все возможные пары).
\end{document}

\documentclass{article}
\usepackage[utf8x]{inputenc}
\usepackage[T1, T2A]{fontenc}
\usepackage[russian]{babel}
\usepackage{amsmath}
\usepackage{amssymb}
\setlength\parindent{0pt}
\usepackage[parfill]{parskip}
\pagenumbering{gobble}

\begin{document}
Разобьем интеграл на слагаемые:
$$\int\limits_\lambda^a \frac{\cos x}{x} dx = \int\limits_\lambda^a \frac{1}{x} dx + \int\limits_\lambda^a \frac{\cos x - 1}{x} dx.$$
Для первого слагаемого имеем:
$$\lim\limits_{\lambda \to 0+} \frac{1}{\ln \lambda} \int\limits_\lambda^a \frac{1}{x} dx = -1.$$
Для второго слагаемого:
$$\left| \int\limits_\lambda^a \frac{\cos x - 1}{x} dx \right| \leqslant \left| \int\limits_0^a \frac{\cos x - 1}{x} dx \right|.$$
Подынтегральная функция непрерывна в нуле (так как $\cos x - 1 = O(x^2)$), поэтому интеграл конечен, откуда получаем:
$$\lim\limits_{\lambda \to 0+} \frac{1}{\ln \lambda} \int\limits_\lambda^a \frac{\cos x - 1}{x} dx = 0.$$
Поскольку оба предела существуют и конечны, предел суммы равен $-1$.
\end{document}

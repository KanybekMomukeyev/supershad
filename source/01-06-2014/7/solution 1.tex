\documentclass{article}
\usepackage[utf8x]{inputenc}
\usepackage[T1, T2A]{fontenc}
\usepackage[russian]{babel}
\usepackage{amsmath}
\usepackage{amssymb}
\setlength\parindent{0pt}
\usepackage[parfill]{parskip}
\pagenumbering{gobble}

\begin{document}
За $O(n^2)$ находим максимум $a$ и второй максимум $b$ в матрице. Если полученные элементы не лежат в одной строке или столбце, то задача решена. Если же лежат --- находим максимальный элемент $c$ из тех элементов, которые не лежат в той строке/столбце, которая общая для $a$ и $b$. Тогда если $c$ не лежит в строке/столбце с $a$, то $ac$ будет искомым максимальным произведением. Если лежит --- то находим максимум $d$ среди тех элементов, которые не лежат в той строке/столбце, которая общая для $a$ и $b$ и не лежат в одной строке/столбце с $a$. Тогда либо $bc$, либо $ad$ будет искомым максимальным произведением.
\end{document}

\documentclass{article}
\usepackage[utf8x]{inputenc}
\usepackage[T1, T2A]{fontenc}
\usepackage[russian]{babel}
\usepackage{amsmath}
\usepackage{amssymb}
\setlength\parindent{0pt}
\usepackage[parfill]{parskip}
\pagenumbering{gobble}

\begin{document}
Рассмотрим функцию от матрицы билинейной формы $f(X)$, которая возвращает максимальное собственное значение. Поскольку матрица симметрична, все собственные значения вещественны, а значит, 
функция определена для любой матрицы. В случае двумерного пространства понятно, что $f(X)$ является непрерывной как композиция непрерывных функций, образованных элементами матрицы. Теперь заметим, 
что $f(A) > 0$, а $f(B) < 0$. Тогда по теореме о промежуточных значениях непрерывной функции $\exists X_0: f(X_0) = 0$. Поскольку у такой матрицы $X_0$ есть собственное значение $0$ --- это вырожденная матрица.
\end{document}

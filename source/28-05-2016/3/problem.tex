\documentclass{article}
\usepackage[utf8x]{inputenc}
\usepackage[T1, T2A]{fontenc}
\usepackage[russian]{babel}
\usepackage{amsmath}
\usepackage{amssymb}
\setlength\parindent{0pt}
\usepackage[parfill]{parskip}
\pagenumbering{gobble}

\begin{document}
Случайные величины $X$ и $Y$ независимы. Плотность случайной величины $X$ равна
$p_X (t) = \frac{t}{2} \cdot I_{[0;2]} (t)$ (где $I_{[0;2]} (t)$ --- индикаторная функция отрезка $[0;2]$), а $Y$ имеет равномерное распределение на отрезке $[0;3]$.
Найдите вероятность того, что из отрезков с длинами $X$, $Y$ и $1$ можно составить треугольник.
\end{document}

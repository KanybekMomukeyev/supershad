\documentclass{article}
\usepackage[utf8x]{inputenc}
\usepackage[T1, T2A]{fontenc}
\usepackage[russian]{babel}
\usepackage{amsmath}
\usepackage{amssymb}
\setlength\parindent{0pt}
\usepackage[parfill]{parskip}
\pagenumbering{gobble}

\begin{document}
Производя рекуррентные подстановки, нетрудно получить явное выражение $$c_n = \frac{1}{n-1} \sum\limits_{k=1}^{n-1} k \beta_{k}.$$
Поскольку последовательность $n^2 \beta_n$ сходится, она ограничена. Более формально $\exists A: \forall k \;\; |\beta_k| \leqslant \frac{A}{k^2}$. Тогда
$$|c_n| \leqslant \frac{1}{n-1} \sum\limits_{k=1}^{n-1} k |\beta_{k}| \leqslant \frac{1}{n-1} \sum_{k=1}^{n-1} \frac{A}{k}.$$
Воспользуемся неравенством между средним арифметическим и средним квадратическим
$$|c_n| \leqslant A \sqrt{\frac{\sum\limits_{k=1}^{n-1} \frac{1}{k^2}}{n-1}} \leqslant A \sqrt{\frac{2}{n-1}}.$$
Поскольку модуль $c_n$ ограничен рядом, сходящимся к $0$, $c_n \to 0$ при $n \to \infty$.
\end{document}

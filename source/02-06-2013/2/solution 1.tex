\documentclass{article}
\usepackage[utf8x]{inputenc}
\usepackage[T1, T2A]{fontenc}
\usepackage[russian]{babel}
\usepackage{amsmath}
\usepackage{amssymb}
\setlength\parindent{0pt}
\usepackage[parfill]{parskip}
\pagenumbering{gobble}

\DeclareMathOperator{\rank}{rank}

\begin{document}
\paragraph{Автор} emazhnik
\paragraph{Решение} Пусть $\{A_j\}_{j=1}^n$ -- столбцы матрицы $A$, а $\{P_k\}$ -- столбцы вида $(1^k, 2^k, \ldots, n^k)^\mathrm{T}$.
Тогда по условию задачи $A_j = P_2 - 2jP_1 + j^2P_0$. Отсюда следует, что столбцы матрицы $A$ принадлежат подпространству, натянутому на линейно независимые вектора $P_2, P_1, P_0$.
Значит, $\rank A \leq 3$.

Заметим, что в случае  $n\geq 3$ столбы $A_1, A_2, A_3$ будут линейно независимы. Действительно,
$$\begin{vmatrix}
1&-2&1\\
1&-4&4\\
1&-6&9
\end{vmatrix}
= -4 \neq 0$$
Поэтому при $n\geq 3$ $\rank A = 3$. В случаях $n=1$ и $n=2$ получим $0$ и $2$ соответственно.
\end{document}


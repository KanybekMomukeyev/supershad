\documentclass{article}
\usepackage[utf8x]{inputenc}
\usepackage[T1, T2A]{fontenc}
\usepackage[russian]{babel}
\usepackage{amsmath}
\usepackage{amssymb}
\setlength\parindent{0pt}
\usepackage[parfill]{parskip}
\pagenumbering{gobble}

\begin{document}
Обозначим исходный массив через $a$. Заведем еще четыре массива длины $n$: $b$, $c$, $d$ и $e$. Пройдем во возрастанию индексов и заполним массив $b$ по правилу $b[0] = 2a[0] - 1$, $b[i] = b[i-1] + 2a[0] - 1$ при $i>1$. Иными словами, если в массиве $a$ заменить нули на $-1$, то в массиве $b$ будут стоять суммы от $a[0]$ до $a[i]$. Заполним массивы $c$ и $d$ минус единицами. Далее идем по возрастанию номеров по массиву $b$. Если $b[i] = k$, то $d[k] = i$, если при этом выполнено $c[k] = -1$, то присваеваем $c[k]=i$.\\
Далее, когда мы прошли по всему массиву $b$, заполним массив $e$ по правилу $e[i] = d[i] - c[i]$. Заметим, что если в массиве $a$ есть подмассив от $a[m]$ до $a[l]$, в котором равное количество единиц и нулей, то $b[m]=b[l]=k$. И тогда $c[k]$ --- это минимальный номер $i$ такой, что $b[i]=k$, а $d[k]$ --- максимальный. Соответственно, $e[k]$ --- максимальное расстояние между $m$ и $l$, где $b[m] = b[l] = k$. Найдем максимум массива $e$, пусть он равен $e[j]$. Тогда искомый подмассив --- это подмассив от $a[c[j]]$ до $a[d[j]]$. 
\end{document}

\documentclass{article}
\usepackage[utf8x]{inputenc}
\usepackage[T1, T2A]{fontenc}
\usepackage[russian]{babel}
\usepackage{amsmath}
\usepackage{amssymb}
\setlength\parindent{0pt}
\usepackage[parfill]{parskip}
\pagenumbering{gobble}

\begin{document}
Идея состоит в том, чтобы рассматривать массив $A$ как подстановку. Пусть индекс $i$ пробегает значения от $0$ до $n-1$. Когда мы встречаем положительный элемент $A[i]$, переходим от него к элементу 
$A[A[i]-1]$, от элемента $A[A[i]-1]$ к элементу $A[A[A[i]-1]-1]$ и так далее, пока мы не не вернемся к $A[i]$, либо не сможем совершить очередной шаг (в таком случае, массив перестановкой не является). В процессе меняем знак всех пройденных элементов на отрицательный. Поскольку на каждом элементе массива мы можем оказаться максимум два раза, итоговая сложность --- $O(n)$. Дополнительная память --- $O(1)$.
\end{document}

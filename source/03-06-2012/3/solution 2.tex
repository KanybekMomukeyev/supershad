\documentclass{article}
\usepackage[utf8x]{inputenc}
\usepackage[T1, T2A]{fontenc}
\usepackage[russian]{babel}
\usepackage{amsmath}
\usepackage{amssymb}
\setlength\parindent{0pt}
\usepackage[parfill]{parskip}
\pagenumbering{gobble}

\begin{document}
Заметим, что такая сумма равна $$\left( 1 + \frac11 \right) \left( 1 + \frac12 \right) \cdots \left( 1 + \frac{1}{n} \right) - 1.$$
В самом деле, в результате раскрытия скобок мы получим все возможные комбинации произведений обратных элементов. Приведем слагаемые в скобках к общему знаменателю. Тогда
$$\frac21 \frac32 \cdots \frac{n+1}{n} - 1 = n+1 - 1 = n.$$
\end{document}

\documentclass{article}
\usepackage[utf8x]{inputenc}
\usepackage[T1, T2A]{fontenc}
\usepackage[russian]{babel}
\usepackage{amsmath}
\usepackage{amssymb}
\setlength\parindent{0pt}
\usepackage[parfill]{parskip}
\pagenumbering{gobble}

\begin{document}
Рассмотрим случайную перестановку $P = (p_1,p_2,\ldots,p_n)$ натуральных чисел от $1$ до $n$. Пару чисел $(i,j)$ назовем <<обменом>>, 
если выполняются соотношения $p_i=j, p_j=i$. Вычислите математическое ожидание количества обменов в перестановке $P$ (перестановка выбирается случайно равновероятно из множества всех перестановок 
от $1$ до $n$).
\end{document}

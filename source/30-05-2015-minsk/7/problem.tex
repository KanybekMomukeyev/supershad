\documentclass{article}
\usepackage[utf8x]{inputenc}
\usepackage[T1, T2A]{fontenc}
\usepackage[russian]{babel}
\usepackage{amsmath}
\usepackage{amssymb}
\setlength\parindent{0pt}
\usepackage[parfill]{parskip}
\usepackage{listings}
\pagenumbering{gobble}

\begin{document}
Рассмотрим четыре реализации одной и той же функции на языке программирования \textit{python}. Определите, что должна вычислять функция. 
Какие из реализаций работают корректно?\\
(a) \begin{lstlisting}
def solve(n, k):
    if n < 0 or k < 0 or k > n: return 0
    if n == 0 or k == 0 or n == k: return 1
    s = 0
    step = k
    if n > 47: step = 10
    for i in range(n + 1):
        s += solve(n - step, i * solve(step, k - i)
    return s
\end{lstlisting}
(b) \begin{lstlisting}
def solve(n, k):
    A = [ 0 for i in range(n+1) ]
    for s in range(16**n)
        tmp = s
        odd = 0
        for t in range(n):
            if tmp % 2: odd += 1
            tmp = tmp // 16
        A[odd] += 1
    return A[k] // 2**(3*n)
\end{lstlisting}
(c) \begin{lstlisting}
def solve(n, k):
    if k == 0 or n == k: return 1
    return solve(n + 1, k) - solve(n, k - 1)
\end{lstlisting}
(d) \begin{lstlisting}
def solve(n, k):
    if k == 0 or n == k: return 1
    return solve(n, k + 1) * (k + 1) // (n - k)
\end{lstlisting}
\textit{Замечание.} Некоторые разъяснения к синтаксису \textit{python}.\\
--- \lstinline!range(x)! возвращает массив \lstinline![0,1,...,x-1]!\\
--- \lstinline!**! возведение в степень, например, \lstinline!2**5 == 32!\\
--- \lstinline!%! взятие остатка от деления, например, \lstinline!7 % 3 == 1!\\
--- \lstinline!//! целочисленное деление, например, \lstinline!7 // 3 == 2!
\end{document}

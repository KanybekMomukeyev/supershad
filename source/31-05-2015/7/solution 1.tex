\documentclass{article}
\usepackage[utf8x]{inputenc}
\usepackage[T1, T2A]{fontenc}
\usepackage[russian]{babel}
\usepackage{amsmath}
\usepackage{amssymb}
\setlength\parindent{0pt}
\usepackage[parfill]{parskip}
\pagenumbering{gobble}

\begin{document}
1. Пусть $A$ обратима. Тогда $$\det (E-AB) = \det (A^{-1}) \det (E-AB) \det A = \det(E - BA).$$
2. Пусть $A$ необратима. Тогда рассмотрим полином
$$P(x) = \det (E - (A - xE)B) - \det(E - B(A - xE)).$$ Если $x$ не является собственным значением $A$, матрица $A-xE$ обратима, поэтому для таких $x$ $P(x) = 0$. Получаем, что $P(x)$ --- полином конечной степени, имеющий бесконечное число корней, а значит, $P(x) \equiv 0$.
\end{document}

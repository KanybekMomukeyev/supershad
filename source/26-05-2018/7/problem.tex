\documentclass{article}
\usepackage[utf8x]{inputenc}
\usepackage[T1, T2A]{fontenc}
\usepackage[russian]{babel}
\usepackage{amsmath}
\usepackage{amssymb}
\setlength\parindent{0pt}
\usepackage[parfill]{parskip}
\pagenumbering{gobble}

\begin{document}
Для двух квадратных матриц $A$ и $B$ одного и того же размера $n$ обозначим через $A \bigstar B$ матрицу, определяемую следующим образом:
$$(A \bigstar B)_{ij} = \begin{cases} (AB)_{ij}, & \text{если $i$ нечетно,} \\ b_{ij}, & \text{иначе.} \end{cases}$$
Для матрицы $A$ определим оператор $\Phi_A : B \mapsto A \bigstar B$ на пространстве матриц $n \times n$.\\
(a) Может ли этот оператор иметь собственное значение $2$ для какой-либо матрицы $A$?\\
(b) Какое наибольшее число различных собственных значений может иметь такой оператор (при фиксированном $n$)?
\end{document}

\documentclass{article}
\usepackage[utf8x]{inputenc}
\usepackage[T1, T2A]{fontenc}
\usepackage[russian]{babel}
\usepackage{amsmath}
\usepackage{amssymb}
\setlength\parindent{0pt}
\usepackage[parfill]{parskip}
\pagenumbering{gobble}

\begin{document}
Получившаяся матрица $A$ будет иметь собственное значение $1$, если матрица $A - E$ будет вырожденной. Добиться этого Ваня может, например, следующим образом. После того, как Дима вписал какой-то элемент $a_{ij}$, Ваня вписывает новый элемент $a_{ik}$ в ту же строку таким образом, чтобы $a_{ik}-\delta_{ik}=-(a_{ij}-\delta_{ij})$, где $\delta_{ij}$ --- символ Кронекера. Тогда сумма чисел в каждой из строк матрицы $A - E$ будет равна нулю, то есть матрица $A - E$ будет вырожденной.
\end{document}

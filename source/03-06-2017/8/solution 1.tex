\documentclass{article}
\usepackage[utf8x]{inputenc}
\usepackage[T1, T2A]{fontenc}
\usepackage[russian]{babel}
\usepackage{amsmath}
\usepackage{amssymb}
\setlength\parindent{0pt}
\usepackage[parfill]{parskip}
\pagenumbering{gobble}

\begin{document}
(а) Исследуем на абсолютную сходимость.\\
Разберем сначала вспомогательное утверждение:
$$\left| \frac{\sin k}{k} \right| + \left| \frac{\sin (k+1)}{k+1} \right| \geqslant \frac{1}{6k},\;\; \forall k \geqslant 1.$$
Рассмотрим два случая:\\
(i) $|\sin k | \geqslant \frac13 \Rightarrow$ утверждение очевидно\\
(ii) $|\sin k| \leqslant \frac13$\\
Рассмотрим единичную окружность и проведем две прямые: $y=\frac13$ и $y=-\frac13$. Тогда множество $|\sin k| \leqslant \frac13$ расположено на окружности между двумя этими прямыми. 
Из геометрических соображений можно увидеть, что при повороте этого множества на $1$ радиан, оно выйдет за пространство между прямыми. Это означает, что $|\sin(k+1)| \geqslant \frac13$ и 
$\left| \frac{\sin(k+1)}{k+1} \right| \geqslant \frac{1}{3(k+1)} \geqslant \frac{1}{6k}$, откуда следует исходное утверждение. В частности, из этого утверждения следует расходимость ряда $\sum\limits_{k=1}^\infty \left| \frac{\sin k}{k} \right|$.\\
Попробуем теперь на основе этой идеи получить оценку на сумму абсолютных значений соседних членов исходного ряда:
$$|a_k| + |a_{k+1}| = \left| \int\limits_0^{\frac{\sin k}{k}} \frac{\sin t}{t} dt \right| + \left| \int\limits_0^{\frac{\sin (k+1)}{k+1}} \frac{\sin t}{t} dt \right| = \int\limits_0^{\left| \frac{\sin k}{k} \right|} \frac{\sin t}{t} dt + \int\limits_0^{\left| \frac{\sin (k+1)}{k+1} \right|} \frac{\sin t}{t} dt.$$
(здесь мы воспользовались тем, что $\frac{\sin x}{x}$ --- четная функция, первый нуль которой расположен в точках $\pm \pi$ и при интегрировании не достигается).
Используя аналогичные рассуждения, получим:
$$|a_k| + |a_{k+1}| \geqslant \int\limits_0^{\frac{1}{6k}} \frac{\sin t}{t} dt \geqslant \sin \frac{1}{6k}.$$
Воспользуемся неравенством $\sin x \geqslant \frac{2x}{\pi}$ при $x \in [0; \frac{\pi}{2}]$. Тогда получим
$$|a_k| + |a_{k+1}| \geqslant \frac{1}{3\pi k},$$
и по признаку сравнения заключаем, что ряд не сходится абсолютно.\\
(б) Исследуем интеграл на условную сходимость.\\
Представим члены ряда в следующем виде:
$$a_k = \int\limits_0^{\frac{\sin k}{k}} \frac{\sin t}{t} dt = \int\limits_0^{\frac{\sin k}{k}} 1 + O(t^2) dt = \frac{\sin k}{k} + O\left( \left( \frac{\sin k}{k} \right)^3 \right).$$
Также заметим, что $O\left( \left( \frac{\sin k}{k} \right)^3 \right)$ сходится, поскольку ряд $\left( \frac{\sin k}{k} \right)^3$ сходится абсолютно. Тогда по признаку сравнения, если ряд $\sum\limits_{k=1}^\infty \frac{\sin k}{k}$ сходится, то сходится и исходный ряд.\\
Воспользуемся признаком Дирихле для сходимости рядов Абелева типа. Последовательность $\{\frac{1}{n}\}$ монотонна и стремится к нулю, осталось доказать ограниченность последовательности 
частичных сумм $\sum\limits_{k=1}^n \sin k$.\\
Поскольку
$$\sum\limits_{k=1}^n \sin k = \frac{\sin(\frac{n}{2}) \sin(\frac{n+1}{2})}{\sin(\frac{1}{2})} \leqslant \frac{1}{\sin (\frac12)},$$
получаем, что ряд сходится условно (чтобы посчитать последнюю сумму, можно домножить ее на $2\sin(\frac12)$ и расписать произведения синусов через разности косинусов).
\end{document}

\documentclass{article}
\usepackage[utf8x]{inputenc}
\usepackage[T1, T2A]{fontenc}
\usepackage[russian]{babel}
\usepackage{amsmath}
\usepackage{amssymb}
\setlength\parindent{0pt}
\usepackage[parfill]{parskip}
\pagenumbering{gobble}

\begin{document}
Поскольку элементы столбцов блочной матрицы коммутируют, характеристическое уравнение можно представить в следующем виде:
$$\det \begin{pmatrix}-\lambda E&-A\\A&-\lambda E\end{pmatrix} = \det(A^2 + \lambda^2 E^2) = \det(A - i \lambda E) \det(A + i \lambda E) = 0$$
Видно, что уравнению удовлетворяют $2n$ чисел $\pm i \lambda$ --- они и будут искомыми собственными значениями.
\end{document}

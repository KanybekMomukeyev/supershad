\documentclass{article}
\usepackage[utf8x]{inputenc}
\usepackage[T1, T2A]{fontenc}
\usepackage[russian]{babel}
\usepackage{amsmath}
\usepackage{amssymb}
\setlength\parindent{0pt}
\usepackage[parfill]{parskip}
\pagenumbering{gobble}

\DeclareMathOperator{\rank}{rank}

\begin{document}
\paragraph{Автор} emazhnik
\paragraph{Решение} Пусть $g(x) = f(x) - f(x-1)$. Тогда
$\begin{cases}
g(1) = f(1) - f(0), \\
g(2) = f(2) - f(1).
\end{cases}$

Заметим, что $g(1) + g(2) = f(2) - f(0) = 0$. Отсюда следует, что функция $g(x)$ меняет знак на отрезке $[1,2]$ (теорема о промежуточном значении непрерывной функции).
Иными словами, $\exists x_0 \in [1,2]: g(x_0)=0$, откуда следует утверждение задачи.
\end{document}

